\documentclass[a4paper,12pt,times,print,index, custombib]{PhDThesisPSnPDF}\usepackage[]{graphicx}\usepackage[]{color}
%% maxwidth is the original width if it is less than linewidth
%% otherwise use linewidth (to make sure the graphics do not exceed the margin)
\makeatletter
\def\maxwidth{ %
  \ifdim\Gin@nat@width>\linewidth
    \linewidth
  \else
    \Gin@nat@width
  \fi
}
\makeatother

\definecolor{fgcolor}{rgb}{0.345, 0.345, 0.345}
\newcommand{\hlnum}[1]{\textcolor[rgb]{0.686,0.059,0.569}{#1}}%
\newcommand{\hlstr}[1]{\textcolor[rgb]{0.192,0.494,0.8}{#1}}%
\newcommand{\hlcom}[1]{\textcolor[rgb]{0.678,0.584,0.686}{\textit{#1}}}%
\newcommand{\hlopt}[1]{\textcolor[rgb]{0,0,0}{#1}}%
\newcommand{\hlstd}[1]{\textcolor[rgb]{0.345,0.345,0.345}{#1}}%
\newcommand{\hlkwa}[1]{\textcolor[rgb]{0.161,0.373,0.58}{\textbf{#1}}}%
\newcommand{\hlkwb}[1]{\textcolor[rgb]{0.69,0.353,0.396}{#1}}%
\newcommand{\hlkwc}[1]{\textcolor[rgb]{0.333,0.667,0.333}{#1}}%
\newcommand{\hlkwd}[1]{\textcolor[rgb]{0.737,0.353,0.396}{\textbf{#1}}}%

\usepackage{framed}
\makeatletter
\newenvironment{kframe}{%
 \def\at@end@of@kframe{}%
 \ifinner\ifhmode%
  \def\at@end@of@kframe{\end{minipage}}%
  \begin{minipage}{\columnwidth}%
 \fi\fi%
 \def\FrameCommand##1{\hskip\@totalleftmargin \hskip-\fboxsep
 \colorbox{shadecolor}{##1}\hskip-\fboxsep
     % There is no \\@totalrightmargin, so:
     \hskip-\linewidth \hskip-\@totalleftmargin \hskip\columnwidth}%
 \MakeFramed {\advance\hsize-\width
   \@totalleftmargin\z@ \linewidth\hsize
   \@setminipage}}%
 {\par\unskip\endMakeFramed%
 \at@end@of@kframe}
\makeatother

\definecolor{shadecolor}{rgb}{.97, .97, .97}
\definecolor{messagecolor}{rgb}{0, 0, 0}
\definecolor{warningcolor}{rgb}{1, 0, 1}
\definecolor{errorcolor}{rgb}{1, 0, 0}
\newenvironment{knitrout}{}{} % an empty environment to be redefined in TeX

\usepackage{alltt}
%\documentclass[a4paper,12pt,times,numbered,print,index]{PhDThesisPSnPDF}

\input{../Preamble/preamble}

%******************************** Glossary *********************************************
\usepackage[toc,acronym]{glossaries}
% suppress page number list in glossary:
%\usepackage[nonumberlist]{glossaries}

\makeglossaries

\renewcommand*{\glstextformat}[1]{\textsf{#1}}
%\renewcommand*{\glshyperlink}[1]{\textsf{#1}}

%--------------
% Glossary
%--------------
\newglossaryentry{fragment}{name=fragment, description={not a PCR duplicate. With paired reads from standard RAD (i. e. including random shearing of restriction fragments) typically identified by having different PE read sequences or different insert sizes after read mapping against a reference}}
\newglossaryentry{RAD tag}{name=RAD tag, description={genetic marker from RAD sequencing; the sequence up or downstream of a restriction site}}
\newglossaryentry{barcode}{name=barcode, description={short DNA sequence incorporated into adapter oligonucleotides that becomes part of the sequence read. Barcodes are used in order to be able to pool the DNA of different individuals, populations, treatments, etc. into one library that can be sequenced on one lane of an illumina flow cell}}
\newglossaryentry{index}{name=index, description={similar to barcode and serves the same purpose; generally incorporated into the centre of the adapter so that special sequencing run for the index is required} }
%\newglossaryentry{SbfI}{name=SbfI, description={restriction enzyme with the recognition sequence \includegraphics[scale=.5]{Sbf-I-cutsite_1_v1_000015}} }
\newglossaryentry{SbfI}{name=SbfI, description={restriction enzyme with the recognition sequence CCTGCA$\downarrow$GG} }
\newglossaryentry{XhoI}{name=XhoI, description={restriction enzyme with the recognition sequence C$\downarrow$TCGAG} }
\newglossaryentry{heterochromatin}{name=heterochromatin, description={Chromatin that remains in a highly condensed state throughout the cell cycle}}
\newglossaryentry{contig}{name=contig, description={longer consensus sequence derived from assembling smaller overlapping sequence reads}}
\newglossaryentry{linked RAD tag site}{name=linked RAD tag site, description={position in the reference sequence with at least one "properly mapped" read pair on either side of a putative restriction site and the SE reads overlapping each other as expected from the restriction enzyme}}
\newglossaryentry{proper pair}{name=proper pair, description={read pair from illumina paired-end sequencing that got mapped to a reference in the correct orientation within a maximum expected distance from each other that is determined by the fragment size selection during the sequencing library preparation}}
\newglossaryentry{kmer}{name=kmer, description={subsequence with a specified length (k) of a longer sequence}}
\newglossaryentry{e-value}{name=Expect (E) value, description={The Expect value (E) is a parameter that describes the number of hits one can "expect" to see by chance when searching a database of a particular size}}
\newglossaryentry{read}{name=read, description={any sequence that comes out of the sequencer}}
\newglossaryentry{edit distance}{name=edit distance, description={minimum number of operations (one symbol insertion, deletion or substitution) required to change one string of symbols into another. Also known as \emph{Levenshtein distance}}}
\newglossaryentry{Ct}{name={C$_{t}$}, description={PCR cycle when a certain fluorescent threshold is reached}}
\newglossaryentry{mqs}{name={mapping quality score}, description={The mapping quality score \emph{Q} is the Phred transformation of the estimate of the probability \emph{p} that the reported mapping position does not correspond to the read's true point of origin: $Q = -10 \log_{10} p$. The way \emph{p} is estimated is different for each mapping programme, but in any case a mapping quality score \emph{Q} of 3 roughly corresponds to a mis-mapping probability \emph{p} of 0.5, i. e. the read has an estimated 50\% chance to have derived from a location other than the one reported}}
\newglossaryentry{discordant}{name=discordant, description={A read pair that aligns with the expected relative mate orientation and with the expected range of distances between mates is said to align "concordantly". This is also called a \gls{proper pair}. If both mates have unique alignments, but the alignments do not match paired-end expectations (i.e. the mates aren't in the expected relative orientation, or aren't within the expected distance range, or both), the pair is said to align "discordantly"}}

%----------------
% Acronyms
%----------------
\newacronym{snp}{SNP}{single nucleotide polymorphism}
\newacronym{rad}{RAD}{Restriction Site associated DNA}
\newacronym{pe}{PE}{paired-end}
\newacronym{se}{SE}{single-end}
\newacronym{bp}{bp}{base pair}
\newacronym{indel}{indel}{sequence insertion or deletion polymorphism}
\newacronym{SAM}{SAM}{Sequence Alignment/Map format}
\newacronym{EST}{EST}{expressed sequence tag}

%%%%%% -- KNITR SETUP -- %%%%%%%%%%

%%%%%%%%%%%%%%%%%%%%%%%%%%
\IfFileExists{upquote.sty}{\usepackage{upquote}}{}
\begin{document}

\printglossaries

\chapter{Testing incomplete digestion}

% **************************** Define Graphics Path **************************
\ifpdf
    \graphicspath{{3_Chapter/Figs/Raster/}{3_Chapter/Figs/PDF/}{3_Chapter/Figs/}}
\else
    \graphicspath{{3_Chapter/Figs/Vector/}{3_Chapter/Figs/}}
\fi

% ************************************************************************************

%\epigraph{Nothing in biology makes sense, except in the light of evolution}{Theodosius Dobzhansky}
%
%\epigraph{\dots it occurred to me to ask the question, Why do some die and some live? And the answer was clearly, that on the whole the best fitted live. [\dots] Then it suddenly flashed upon me that this self-acting process would necessarily improve the race, because in every generation the inferior would inevitably be killed off and the superior would remain-that is, the fittest would survive \dots}{Alfred Russel Wallace}

\epigraph{
Bonzai: Are you as successful as you would like to be?

Zappa: I would say that the basic characteristic of my life is failure. 
If there is one thing that I excel at, it's failure -- I manage to fail at 100 percent of the things that I do. 
Since most of the things that I set out to do are theoretically impossible, it's very easy to fail. 
I've learned to live with it. 
In terms of machinery and personnel, there never seems to be enough to get things done exactly right.
}{interview with Frank Zappa, 1985}

\section{The Problem}

describe the observations in the stacks output
incomplete digestion as a possible underlying cause

\section{Data and Methods}

\section{Results}

\section{Interpretation of Results}



% ********************************************************
And now I begin my third chapter here \dots

\cite{Baird2008} were the first to publish about RAD. \SI{12,3}{\micro\metre}

Fig.~\vref{fragments-mapped-per-ind} shows the number of \glspl{fragment} mapped per individual.
I am trying to find \glspl{snp} among individuals of the sample.
% ********************************************************

\begin{figure}
\begin{knitrout}
\definecolor{shadecolor}{rgb}{0.969, 0.969, 0.969}\color{fgcolor}

{\centering \includegraphics[width=\linewidth]{figure/fragments_mapped_per_ind-1} 

}



\end{knitrout}
\caption{Distribution of RAD fragment numbers mapped to 4 primer3ready reference contigs. 
A "fragment" is a properly mapped read pair from an individual with a unique insert size. 
If two read pairs on a RAD site from an individual have the same insert size, they constitute only one fragment, i. e. one read pair is likely to be a PCR duplicate.}
\label{fragments-mapped-per-ind}
\end{figure}





\subsection{First subsection in the first section}
\dots and some more 

\subsection{Second subsection in the first section}
\dots and some more \dots

\subsubsection{First subsub section in the second subsection}
\dots and some more in the first subsub section otherwise it all looks the same
doesn't it? well we can add some text to it \dots

\subsection{Third subsection in the first section}
\dots and some more \dots

\subsubsection{First subsub section in the third subsection}
\dots and some more in the first subsub section otherwise it all looks the same
doesn't it? well we can add some text to it and some more and some more and
some more and some more and some more and some more and some more \dots

\subsubsection{Second subsub section in the third subsection}
\dots and some more in the first subsub section otherwise it all looks the same
doesn't it? well we can add some text to it \dots

\section{Second section of the third chapter}
and here I write more \dots

\section{The layout of formal tables}
This section has been modified from ``Publication quality tables in \LaTeX*''
 by Simon Fear.

The layout of a table has been established over centuries of experience and 
should only be altered in extraordinary circumstances. 

When formatting a table, remember two simple guidelines at all times:

\begin{enumerate}
  \item Never, ever use vertical rules (lines).
  \item Never use double rules.
\end{enumerate}

These guidelines may seem extreme but I have
never found a good argument in favour of breaking them. For
example, if you feel that the information in the left half of
a table is so different from that on the right that it needs
to be separated by a vertical line, then you should use two
tables instead. Not everyone follows the second guideline:

There are three further guidelines worth mentioning here as they
are generally not known outside the circle of professional
typesetters and subeditors:

\begin{enumerate}\setcounter{enumi}{2}
  \item Put the units in the column heading (not in the body of
          the table).
  \item Always precede a decimal point by a digit; thus 0.1
      {\em not} just .1.
  \item Do not use `ditto' signs or any other such convention to
      repeat a previous value. In many circumstances a blank
      will serve just as well. If it won't, then repeat the value.
\end{enumerate}

A frequently seen mistake is to use `\textbackslash begin\{center\}' \dots `\textbackslash end\{center\}' inside a figure or table environment. This center environment can cause additional vertical space. If you want to avoid that just use `\textbackslash centering'


\begin{table}
\caption{A badly formatted table}
\centering
\label{table:bad_table}
\begin{tabular}{|l|c|c|c|c|}
\hline 
& \multicolumn{2}{c}{Species I} & \multicolumn{2}{c|}{Species II} \\ 
\hline
Dental measurement  & mean & SD  & mean & SD  \\ \hline 
\hline
I1MD & 6.23 & 0.91 & 5.2  & 0.7  \\
\hline 
I1LL & 7.48 & 0.56 & 8.7  & 0.71 \\
\hline 
I2MD & 3.99 & 0.63 & 4.22 & 0.54 \\
\hline 
I2LL & 6.81 & 0.02 & 6.66 & 0.01 \\
\hline 
CMD & 13.47 & 0.09 & 10.55 & 0.05 \\
\hline 
CBL & 11.88 & 0.05 & 13.11 & 0.04\\ 
\hline 
\end{tabular}
\end{table}

\begin{table}
\caption{A nice looking table}
\centering
\label{table:nice_table}
\begin{tabular}{l c c c c}
\hline 
\multirow{2}{*}{Dental measurement} & \multicolumn{2}{c}{Species I} & \multicolumn{2}{c}{Species II} \\ 
\cline{2-5}
  & mean & SD  & mean & SD  \\ 
\hline
I1MD & 6.23 & 0.91 & 5.2  & 0.7  \\

I1LL & 7.48 & 0.56 & 8.7  & 0.71 \\

I2MD & 3.99 & 0.63 & 4.22 & 0.54 \\

I2LL & 6.81 & 0.02 & 6.66 & 0.01 \\

CMD & 13.47 & 0.09 & 10.55 & 0.05 \\

CBL & 11.88 & 0.05 & 13.11 & 0.04\\ 
\hline 
\end{tabular}
\end{table}


\begin{table}
\caption{Even better looking table using booktabs}
\centering
\label{table:good_table}
\begin{tabular}{l c c c c}
\toprule
\multirow{2}{*}{Dental measurement} & \multicolumn{2}{c}{Species I} & \multicolumn{2}{c}{Species II} \\ 
\cmidrule{2-5}
  & mean & SD  & mean & SD  \\ 
\midrule
I1MD & 6.23 & 0.91 & 5.2  & 0.7  \\

I1LL & 7.48 & 0.56 & 8.7  & 0.71 \\

I2MD & 3.99 & 0.63 & 4.22 & 0.54 \\

I2LL & 6.81 & 0.02 & 6.66 & 0.01 \\

CMD & 13.47 & 0.09 & 10.55 & 0.05 \\

CBL & 11.88 & 0.05 & 13.11 & 0.04\\ 
\bottomrule
\end{tabular}
\end{table}

\bibliographystyle{elsarticle-harv}
\bibliography{/Users/Claudius/Documents/MyLiterature/Literature}

\end{document}
