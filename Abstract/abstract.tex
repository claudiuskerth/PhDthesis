% ************************** Thesis Abstract *****************************
% Use `abstract' as an option in the document class to print only the titlepage and the abstract.
\begin{abstract}
\gls{rad} is a molecular method involving restriction digestion and high throughput DNA sequencing. It promises the systematic subsampling of the genome and highly repeatable scoring of genetic variation in hundreds of individuals at current sequencing costs. However, it comes with its own problems. De novo assembly of \gls{rad} sequence data usually creates many putative reference tags that are only found in one or a few individuals leaving only relatively few markers for population genomic analyses. I here investigate three potential reasons for this outcome -- incomplete digestion, genomic religation and insufficient DNA template amount -- by looking at the occurrence of restriction enzyme recognition sequences within the resultant sequencing data of two different types of \gls{rad} libraries. 

Analysis of sequence clusters as well as the proportion of concordantly mapping read pairs against a \textit{Locusta} reference sequence suggest that incomplete digestion has affected one of the restriction enzymes used and thereby the amount of loci that could be sequenced at sufficient coverage across individuals. The other restriction enzyme is found to be much less affected by incomplete digestion and instead random religation of restriction fragments indicates an inefficient adapter ligation step that also leads to low read coverage across individuals. Finally, qPCR and read mapping against four newly reconstructed \gls{pe} contig pair reference sequences suggests that low amount of starting DNA and/or high loss of DNA during the library preparation are a major cause for the locus drop out observed in the de novo assembled read data.

In the second part of this thesis I am using \gls{rad} sequence data to make inferences about several aspects of the demographic history of two grasshopper subspecies. Sequence data was generated from 36 individuals sampled at the two opposite ends of a hybrid zone of two grasshopper subspecies that is characterised by hybrid male sterility. I use a state--of--the--art de novo assembly strategy that utilises the shotgun--type \gls{pe} reads from standard \gls{rad} to distinguish alleles from paralogs. I then conduct several population genomic analyses with the programme \texttt{ANGSD} that incorporates uncertainty in genotypes by using genotype likelihoods instead of called genotypes. Results based on more than 1 million filtered sites show the strong genetic differentiation of the two subspecies and a surprisingly high genetic diversity in the subspecies that is thought to be derived from a very distant glacial refuge. Further analysis of this data set promises to yield more insights.
\end{abstract}
